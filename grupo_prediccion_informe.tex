% Options for packages loaded elsewhere
\PassOptionsToPackage{unicode}{hyperref}
\PassOptionsToPackage{hyphens}{url}
%
\documentclass[
  12pt,
]{article}
\usepackage{amsmath,amssymb}
\usepackage{iftex}
\ifPDFTeX
  \usepackage[T1]{fontenc}
  \usepackage[utf8]{inputenc}
  \usepackage{textcomp} % provide euro and other symbols
\else % if luatex or xetex
  \usepackage{unicode-math} % this also loads fontspec
  \defaultfontfeatures{Scale=MatchLowercase}
  \defaultfontfeatures[\rmfamily]{Ligatures=TeX,Scale=1}
\fi
\usepackage{lmodern}
\ifPDFTeX\else
  % xetex/luatex font selection
\fi
% Use upquote if available, for straight quotes in verbatim environments
\IfFileExists{upquote.sty}{\usepackage{upquote}}{}
\IfFileExists{microtype.sty}{% use microtype if available
  \usepackage[]{microtype}
  \UseMicrotypeSet[protrusion]{basicmath} % disable protrusion for tt fonts
}{}
\makeatletter
\@ifundefined{KOMAClassName}{% if non-KOMA class
  \IfFileExists{parskip.sty}{%
    \usepackage{parskip}
  }{% else
    \setlength{\parindent}{0pt}
    \setlength{\parskip}{6pt plus 2pt minus 1pt}}
}{% if KOMA class
  \KOMAoptions{parskip=half}}
\makeatother
\usepackage{xcolor}
\usepackage[margin=1in]{geometry}
\usepackage{graphicx}
\makeatletter
\def\maxwidth{\ifdim\Gin@nat@width>\linewidth\linewidth\else\Gin@nat@width\fi}
\def\maxheight{\ifdim\Gin@nat@height>\textheight\textheight\else\Gin@nat@height\fi}
\makeatother
% Scale images if necessary, so that they will not overflow the page
% margins by default, and it is still possible to overwrite the defaults
% using explicit options in \includegraphics[width, height, ...]{}
\setkeys{Gin}{width=\maxwidth,height=\maxheight,keepaspectratio}
% Set default figure placement to htbp
\makeatletter
\def\fps@figure{htbp}
\makeatother
\setlength{\emergencystretch}{3em} % prevent overfull lines
\providecommand{\tightlist}{%
  \setlength{\itemsep}{0pt}\setlength{\parskip}{0pt}}
\setcounter{secnumdepth}{-\maxdimen} % remove section numbering
\usepackage{graphicx}
\usepackage{fancyhdr}
\usepackage{mdframed}
\pagestyle{fancy}
\fancyhead{}
\fancyhead[L]{\includegraphics[width=3cm]{logo.jpg}}
\fancyhead[R]{\hspace{0.5cm} Universidad Diego Portales \\ Facultad de Administración y Economía}
\fancypagestyle{plain}{ \fancyhead{} \fancyhead[L]{\includegraphics[width=3cm]{logo.jpg}} \fancyhead[R]{\hspace{0.5 cm} Universidad Diego Portales \\ Facultad de Administración y Economía}}
\usepackage{booktabs}
\usepackage{longtable}
\usepackage{array}
\usepackage{multirow}
\usepackage{wrapfig}
\usepackage{float}
\usepackage{colortbl}
\usepackage{pdflscape}
\usepackage{tabu}
\usepackage{threeparttable}
\usepackage{threeparttablex}
\usepackage[normalem]{ulem}
\usepackage{makecell}
\usepackage{xcolor}
\ifLuaTeX
  \usepackage{selnolig}  % disable illegal ligatures
\fi
\usepackage{bookmark}
\IfFileExists{xurl.sty}{\usepackage{xurl}}{} % add URL line breaks if available
\urlstyle{same}
\hypersetup{
  pdftitle={¿Cuál será la oferta de dinero M2 en Chile el próximo trimestre basada en las variables macroeconómicas disponibles?},
  hidelinks,
  pdfcreator={LaTeX via pandoc}}

\title{\textbf{¿Cuál será la oferta de dinero M2 en Chile el próximo
trimestre basada en las variables macroeconómicas disponibles?}}
\author{}
\date{\vspace{-2.5em}}

\begin{document}
\maketitle

\maketitle
\vspace{4cm}
\begin{center}
\large \textbf{Estudiantes:} \\
\vspace{0.5cm}
\large Dania Bustamante \\
\large Rosana Cardona \\
\large José Casanova \\
\vfill
\large \textbf{Curso}: Econometría II \\
\large \textbf{Profesor}: Mauricio Tejada \\
\large 3 julio 2024 \\
\vspace{1cm}
\end{center}
\newpage
\tableofcontents
\newpage

\section{Introducción}\label{introducciuxf3n}

La oferta monetaria M2 es un componente esencial en el análisis
económico, ya que incluye tanto el dinero en circulación como otros
depósitos de alta liquidez. La predicción de M2 es fundamental para la
formulación de políticas económicas, especialmente en contextos de
estabilidad financiera y control de la inflación. Este estudio plantea
la pregunta: ¿Cuál será la oferta de dinero M2 en Chile el próximo
trimestre basada en las variables macroeconómicas disponibles? La
hipótesis es que variables como la tasa de política monetaria, el tipo
de cambio nominal, el índice de precios al consumidor (IPC) y el PIB
tienen un impacto significativo en la predicción de M2.

Investigaciones previas han mostrado una alta correlación entre la
oferta monetaria, el PIB y los niveles de precios. Por ejemplo, Friedman
y Schwartz (1963) demostraron que la oferta monetaria está estrechamente
relacionada con el PIB y la inflación. Bernanke y Blinder (1992) también
subrayaron la importancia de la política monetaria en la determinación
de la oferta monetaria.

Para abordar esta pregunta, este estudio incluye variables clave como la
tasa de política monetaria, el tipo de cambio nominal, el IPC y el PIB,
con el fin de entender cómo influyen en la oferta monetaria y, por ende,
en la economía en general. La inclusión de estas variables permite un
análisis detallado de los factores que afectan M2 y cómo las políticas
monetarias pueden ser ajustadas para alcanzar objetivos económicos
específicos. \newpage

\section{Metodología}\label{metodologuxeda}

\subsection{Definición del Modelo y Método de
Estimación:}\label{definiciuxf3n-del-modelo-y-muxe9todo-de-estimaciuxf3n}

Para responder a nuestra pregunta de investigación, utilizamos dos
enfoques principales: el Modelo Autoregresivo de Retardos Distribuidos
(ARD) y el Modelo Vector Autoregresivo (VAR). Cada uno de estos modelos
tiene sus ventajas y consideraciones, utilizando ambos para obtener una
visión integral de los factores que influyen en la oferta monetaria M2 y
la capacidad de estos modelos para realizar predicción.

\subsection{Modelo Autoregresivo de Retardos Distribuidos
(ARD)}\label{modelo-autoregresivo-de-retardos-distribuidos-ard}

El Modelo Autoregresivo de Retardos Distribuidos (ARD) nos permite
incluir los rezagos de la variable dependiente (M2) y otras variables
explicativas (IPC, PIB, TCN, TPM), junto con variables dummy para
eventos específicos y estacionalidad. Este enfoque es útil para capturar
la dinámica temporal de la serie de tiempo, permitiendo que los valores
pasados de M2 y otras variables macroeconómicas influyan en los valores
presentes.

La especificación general del modelo ARD es:

\begin{align*}
\Delta \log(\text{M2}_t) &= \beta_0 + \beta_1 \Delta \log(\text{M2}_{t-1}) + \beta_2 \Delta \log(\text{IPC}_t) \\
&\quad + \beta_3 \Delta \log(\text{PIB}_t) + \beta_4 \Delta \log(\text{TCN}_t) + \beta_5 \Delta \log(\text{TPM}_t) \\
&\quad + \beta_6 \delta_{2008} + \beta_7 \delta_{2010} + \beta_8 \delta_{2020} \\
&\quad + \beta_9 \delta_{Q1} + \beta_{10} \delta_{Q2} + \beta_{11} \delta_{Q3} + \epsilon
\end{align*}

En este modelo, las variables dummy permiten capturar efectos
específicos de ciertos eventos históricos y estacionales que podrían
afectar la oferta monetaria M2. Sin embargo, una de las limitaciones del
modelo ARD es la posible existencia de endogeneidad, lo que puede sesgar
las estimaciones y hacer que los resultados sean menos confiables.

En el contexto del modelo ARD, la endogeneidad puede surgir debido a la
relación simultánea entre las variables macroeconómicas. Por ejemplo,
cambios en la tasa de política monetaria pueden afectar el PIB y, a su
vez, el PIB puede influir en la tasa de política monetaria. Esta
interdependencia no se captura completamente en un modelo ARD, lo que
puede llevar a problemas de estimación. Por esta razón, complementamos
nuestro análisis con un modelo VAR, que es más adecuado para manejar la
endogeneidad.

\subsection{Modelo Vector Autoregresivo
(VAR)}\label{modelo-vector-autoregresivo-var}

El Modelo Vector Autoregresivo (VAR) nos permite capturar las
interrelaciones entre todas las variables macroeconómicas incluidas. En
un modelo VAR, todas las variables se trataron como endógenas, y cada
una de ellas se modela como una función de sus propios rezagos y los
rezagos de las demás variables del sistema.

La forma reducida en notación matricial del sistema VAR(1) es:

\[\mathbf{Z}_t = \mathbf{A} + \mathbf{\Phi} \mathbf{Z}_{t-1} + \mathbf{\epsilon}_t\]

Donde:

\begin{itemize}
\tightlist
\item
  \(\mathbf{Z}_t\) es el vector de las variables endógenas en el tiempo
  \(t\),
\item
  \(\mathbf{A}\) es el vector de constantes,
\item
  \(\mathbf{\Phi}\) es la matriz de coeficientes de los rezagos,
\item
  \(\mathbf{\epsilon}_t\) es el vector de términos de error.
\end{itemize}

\[
\begin{pmatrix}
\log(\text{M2}_t) \\
\log(\text{IPC}_t) \\
\log(\text{PIB}_t) \\
\log(\text{TCN}_t) \\
\log(\text{TPM}_t)
\end{pmatrix}
=
\begin{pmatrix}
\alpha_{1} \\
\alpha_{2} \\
\alpha_{3} \\
\alpha_{4} \\
\alpha_{5}
\end{pmatrix}
+
\begin{pmatrix}
\phi_{11} & \phi_{12} & \phi_{13} & \phi_{14} & \phi_{15} \\
\phi_{21} & \phi_{22} & \phi_{23} & \phi_{24} & \phi_{25} \\
\phi_{31} & \phi_{32} & \phi_{33} & \phi_{34} & \phi_{35} \\
\phi_{41} & \phi_{42} & \phi_{43} & \phi_{44} & \phi_{45} \\
\phi_{51} & \phi_{52} & \phi_{53} & \phi_{54} & \phi_{55}
\end{pmatrix}
\begin{pmatrix}
\log(\text{M2}_{t-1}) \\
\log(\text{IPC}_{t-1}) \\
\log(\text{PIB}_{t-1}) \\
\log(\text{TCN}_{t-1}) \\
\log(\text{TPM}_{t-1})
\end{pmatrix}
+
\begin{pmatrix}
\epsilon_{1t} \\
\epsilon_{2t} \\
\epsilon_{3t} \\
\epsilon_{4t} \\
\epsilon_{5t}
\end{pmatrix}
\]

En este modelo, cada ecuación representa una variable endógena y su
relación con los rezagos de todas las variables del sistema. Esto
permite capturar la endogeneidad y las interrelaciones dinámicas entre
las variables macroeconómicas.

El modelo VAR ofrece varias ventajas importantes:

\begin{itemize}
\tightlist
\item
  \textbf{Causalidad de Granger:} Permite evaluar si una variable es
  útil para predecir otra variable a través del análisis de causalidad
  de Granger.
\item
  \textbf{Funciones de Impulso Respuesta:} Permite analizar cómo un
  choque en una variable afecta a las demás variables en el tiempo.
\item
  \textbf{Descomposición de la Varianza:} Permite descomponer la
  variabilidad de cada variable en función de los choques en todas las
  variables del sistema.
\end{itemize}

El uso combinado de los modelos ARD y VAR nos permite abordar de manera
integral la predicción de la oferta monetaria M2, considerando tanto la
dinámica temporal como las interrelaciones entre las variables
macroeconómicas, mitigando los problemas de endogeneidad que podrían
afectar la confiabilidad de los resultados.

\subsection{Evaluación de la
Endogeneidad}\label{evaluaciuxf3n-de-la-endogeneidad}

Para evaluar la endogeneidad de las variables en el modelo VAR, es
importante considerar la teoría económica y cómo cada variable responde
a los cambios en las demás variables. Basado en la literatura económica,
podemos organizar las variables de la más endógena a la menos endógena
de la siguiente manera:

\begin{enumerate}
\def\labelenumi{\arabic{enumi}.}
\tightlist
\item
  \textbf{Oferta Monetaria M2:} Altamente endógena porque responde
  directamente a las políticas del banco central (TPM) y a cambios en el
  nivel de precios (IPC), el PIB y el tipo de cambio nominal (TCN).
\item
  \textbf{Índice de Precios al Consumidor (IPC):} Altamente endógeno
  porque se ve afectado por la oferta monetaria (M2) y el tipo de cambio
  nominal (TCN).
\item
  \textbf{Producto Interno Bruto (PIB):} Endógeno porque responde a
  cambios en la política monetaria (TPM), el nivel de precios (IPC) y la
  oferta monetaria (M2).
\item
  \textbf{Tipo de Cambio Nominal (TCN):} Menor endógeneidad que las
  variables anteriores, sin embargo, responde a cambios en la política
  monetaria (TPM) y a los movimientos del PIB y la oferta monetaria
  (M2).
\item
  \textbf{Tasa de Política Monetaria (TPM):} Poca endógeneidad en
  comparación con las otras variables, ya que es una herramienta de
  política controlada por el banco central.
\end{enumerate}

Basado en el análisis teórico, la oferta monetaria M2 es la variable más
endógena, seguida por el IPC, el PIB, el TCN y finalmente la TPM.

\section{Descripción de los Datos}\label{descripciuxf3n-de-los-datos}

Los datos utilizados en este estudio fueron obtenidos de la página del
Banco Central de Chile, y comprenden varias variables económicas clave
ajustadas a datos trimestrales:

\begin{itemize}
\tightlist
\item
  \textbf{M2:} Oferta monetaria.
\item
  \textbf{IPC:} Índice de Precios al Consumidor, ajustado a trimestral.
\item
  \textbf{PIB:} Producto Interno Bruto.
\item
  \textbf{TCN:} Tipo de Cambio Nominal.
\item
  \textbf{TPM:} Tasa de Política Monetaria.
\end{itemize}

Estas variables proporcionan una visión integral de diferentes aspectos
de la economía chilena, permitiendo análisis detallados sobre el
comportamiento y las interacciones entre políticas monetarias, precios,
actividad económica y el mercado cambiario. Por otra parte, las series
de tiempo se transformaron a logaritmos y se diferenciaron para asegurar
la estacionariedad. Además, se crearon variables dummy para eventos
específicos (2008, 2010, 2020) y estacionales (Q1, Q2, Q3).

\textbf{\emph{Estadísticas Descriptivas:}}

\begingroup\fontsize{10}{12}\selectfont

\begin{longtable}[t]{lrrrrr}
\toprule
 & mean & sd & median & min & max\\
\midrule
M2 & 7.552937e+04 & 5.713168e+04 & 56475.081500 & 10808.9600000 & 1.986452e+05\\
IPC & 3.233038e-01 & 2.901452e-01 & 0.300000 & -0.2666667 & 1.166667e+00\\
PIB & 3.619889e+04 & 1.006787e+04 & 35968.871896 & 19581.2485453 & 5.433831e+04\\
TCN & 6.028063e+02 & 1.274393e+02 & 578.266825 & 408.1122951 & 9.455024e+02\\
TPM & 4.526827e+00 & 2.567536e+00 & 4.540909 & 0.5000000 & 1.125000e+01\\
\bottomrule
\end{longtable}
\endgroup{}

\begin{itemize}
\item
  La Oferta Monetaria M2 (en miles de millones de pesos) posee una alta
  variación dada su desviación estándar. Mediana bien alejada de la
  media, por lo que se asume sesgo por defecto en la distribución.
\item
  El Índice de Precios al Consumidor IPC posee una media cercana a 0,
  variaciones dentro de lo normal, a primera vista no existe sesgo dada
  la naturaleza de la media y la mediana.
\item
  El Producto Interno Bruto PIB (miles de millones de pesos) posee una
  distribución menos fluctuante que las anteriores, la media y la
  mediana se mantienen relativamente parecidas.
\item
  El Tipo de cambio nominal TCN (dólar observado \$CLP/USD) tiene una
  variación que ha ido fluctuando con mayores diferencias en los últimos
  años ya que la mediana es alrededor de 30 pesos más baja que la media.
  Aparentemente una desviación alta.
\item
  La Tasa de política monetaria TPM (porcentaje) tiene una media y
  mediana prácticamente iguales, por no decir la misma, desviación
  estándar considerable dada la naturaleza de los datos.
\end{itemize}

\section{Análisis gráfico}\label{anuxe1lisis-gruxe1fico}

Los gráficos presentados proporcionan una visión integral de la
transformación de las variables económicas para asegurar la
estacionariedad, una condición necesaria para la aplicación de modelos
econométricos. El análisis confirma que, tras la diferenciación, las
series temporales de M2, IPC, PIB, TCN y TPM se han convertido en
estacionarias, lo que valida el uso de modelos ARD y VAR en nuestro
estudio para predecir la oferta monetaria M2. Estos modelos nos permiten
capturar las dinámicas temporales y las interrelaciones entre las
variables, ofreciendo una base sólida para el análisis.

\subsection{Aplicación de logaritmo}\label{aplicaciuxf3n-de-logaritmo}

\subsubsection{Logaritmo de las
variables}\label{logaritmo-de-las-variables}

Al tomar los logaritmos de estas variables, buscamos estabilizar la
varianza y obtener una relación más lineal entre las variables. Los
logaritmos ayudan a manejar la heterocedasticidad y permiten una mejor
interpretación de las tasas de crecimiento.

\begin{center}\includegraphics{grupo_prediccion_informe_files/figure-latex/unnamed-chunk-3-1} \end{center}

Al principio, se observan los logaritmos de las variables con el fin de
estabilizar la varianza y buscar linealidad entre variables. Es
plausible encontrar una tendencia, ya sea creciente o decreciente,
dependiendo de la variable y queda en evidencia la no estacionariedad de
las series de tiempo.

\subsubsection{Autocorrelación de Logaritmo de las
variables}\label{autocorrelaciuxf3n-de-logaritmo-de-las-variables}

El siguietne gráfico presenta las funciones de autocorrelación (ACF) de
las series logarítmicas de las variables económicas. La ACF muestra la
correlación de una variable consigo misma en diferentes rezagos.

\begin{center}\includegraphics{grupo_prediccion_informe_files/figure-latex/unnamed-chunk-4-1} \end{center}

Los gráficos ACF de M2 y PIB, en particular, exhiben un patrón de
autocorrelación que decrece lentamente, lo que confirma la no
estacionariedad de estas series, puesto que se refleja en la
persistencia de la correlación en varios rezagos, sugiriendo que los
valores actuales están influenciados por valores pasados durante un
período prolongado.

\subsection{Aplicación de
diferenciación}\label{aplicaciuxf3n-de-diferenciaciuxf3n}

Para eliminar la tendencia, se diferencian las variables con el fin
convertir las series a estacionarias, así se pueden captar relaciones
significativas entre las variables.

\subsubsection{Diferenciación del Logaritmo de las
variables}\label{diferenciaciuxf3n-del-logaritmo-de-las-variables}

Al realizar la diferenciación, es posible observar que las tendencias a
largo plazo ya desaparecieron, ahora cada serie de tiempo fluctúa en
base a una media constante, se cumple la estacionariedad que tanto se
buscaba.

\begin{center}\includegraphics{grupo_prediccion_informe_files/figure-latex/unnamed-chunk-5-1} \end{center}

\subsubsection{Autocorrelación de la Diferenciación del Logaritmo de las
variables}\label{autocorrelaciuxf3n-de-la-diferenciaciuxf3n-del-logaritmo-de-las-variables}

Las funciones de autocorrelación de las diferencias logarítmicas
confirman que las series se han convertido en estacionarias, ya que la
mayoría de los coeficientes de autocorrelación caen rápidamente a cero,
y no se observa una correlación significativa en los rezagos más
lejanos. Esto es crucial, ya que si se va a trabajar con los modelos ARD
y VAR, es necesario que las series sean estacionarias.

\begin{center}\includegraphics{grupo_prediccion_informe_files/figure-latex/unnamed-chunk-6-1} \end{center}

\section{Resultados del Modelo}\label{resultados-del-modelo}

\subsection{Estimación del Modelo
ARD:}\label{estimaciuxf3n-del-modelo-ard}

El modelo ARD se ajustó con el número óptimo de rezagos basado en el
criterio AIC. Los resultados mostraron que las variables dIPC\_ts y
dPIB\_ts tienen un impacto significativo en dM2\_ts, con coeficientes
positivos y significativos, mientras que las demás variables no
mostraron significancia estadística considerable.

\subsubsection{Evaluación de la Autocorrelación y
Heterocedasticidad:}\label{evaluaciuxf3n-de-la-autocorrelaciuxf3n-y-heterocedasticidad}

\begin{itemize}
\tightlist
\item
  Durbin-Watson Test: No se encontró autocorrelación significativa en
  los residuos.
\item
  Breusch-Pagan Test: No se encontró evidencia significativa de
  heterocedasticidad.
\end{itemize}

Los resultados del modelo ARD, en resumen, son básicamente todos no
significativos, menos el PIB, pero una sola variable significativa no es
suficiente como para quedarse con el modelo, no vale la pena. Esto
debido a que el modelo probablemente no es lo suficientemente robusto y
además es poco complejo para analizar la correlación que puede existir
entre todas las variables estudiadas. Además comentar que el R cuadrado
arroja un resultado bajísimo de apenas 0.16 aproximadamente.

Lo anterior da pie a que se evalúe un modelo VAR, para capturar la
complejidad del problema, donde es probable que incluso se tenga que
realizar una evaluación de causalidad de Granger para saber si las
variables explican en una dirección o si mantienen una doble implicancia
entre ellas.

\subsection{Estimación del Modelo
VAR}\label{estimaciuxf3n-del-modelo-var}

El modelo VAR se ajustó con las variables exógenas y el número óptimo de
rezagos seleccionado.

Según nuestros resultados, aquellas variables que tienen una causalidad
de granger significativa sobre la M2 es principalmente el IPC y la TCN,
mientras que los parámetros PIB y TPM mantienen una causalidad de
granger no significativa.

Esto no quiere decir que el modelo sea incorrecto, esto lo único que
explica es que unas variables mantienen una relación más fuerte para
poder predecir la variable en cuestión. Esto se puede deber en base a la
naturaleza del modelo, o que existe un valor predictivo, solo que no tan
importante versus las otras variables que sí lo son.

Esto incluso en un futuro, podría ayudar a perfeccionar aún más la
predicción, sin estar errónea la predicción que ya existe del M2, solo
que cabe la posibilidad de que sea menos precisa de lo que se esperaba
según el modelo VAR que por defecto si daba resultados significativos.

\subsubsection{Funciones de Impulso
Respuesta:}\label{funciones-de-impulso-respuesta}

Las funciones de impulso respuesta indicaron que un choque en dM2 tiene
efectos persistentes en sí mismo, mientras que choques en dIPC y dPIB
también afectan significativamente a dM2 en el tiempo, confirmando la
importancia de estas variables en la predicción de M2.

En el caso de la M2 sobre sí misma, inicialmente muestra una respuesta
positiva que se disipa rápidamente en los primeros períodos, concluyendo
en una oscilación constante alrededor de la media 0. Estos son choques
transitorios.

\begin{center}\includegraphics{grupo_prediccion_informe_files/figure-latex/unnamed-chunk-7-1} \end{center}

En cuanto a la IPC sobre la M2, se observa un oscilamiento constante
respecto a la media 0 desde el inicio, indicando efectos positivos y
negativos variables. A partir del noveno período, predomina un impulso
mayormente negativo.

\begin{center}\includegraphics{grupo_prediccion_informe_files/figure-latex/unnamed-chunk-8-1} \end{center}

Al observar el PIB sobre la M2, en su mayoría se observa un estímulo
positivo sin un comportamiento oscilatorio como en los casos anteriores.

\begin{center}\includegraphics{grupo_prediccion_informe_files/figure-latex/unnamed-chunk-9-1} \end{center}

En contraste, la TCN tiene efectos casi opuestos sobre la M2, mayormente
negativos y sin un patrón oscilatorio claro, con solo dos casos donde
superó ligeramente la barrera negativa.

\begin{center}\includegraphics{grupo_prediccion_informe_files/figure-latex/unnamed-chunk-10-1} \end{center}

Finalmente, la TPM ejerce un fuerte estímulo positivo en los primeros
períodos, aunque no desde el inicio como tal, cambiando abruptamente
hacia un comportamiento negativo más equilibrado posteriormente.

\begin{center}\includegraphics{grupo_prediccion_informe_files/figure-latex/unnamed-chunk-11-1} \end{center}

\subsubsection{Descomposición de la
Varianza:}\label{descomposiciuxf3n-de-la-varianza}

Las descomposiciones de la varianza, las observamos en las distintas
variables que se están estudiando, y es una forma de asignarle
explicación según qué otra variable estudiada, es decir, qué tanto se
explican entre sí. Lo lógico sería que a medida que pasa el tiempo, lo
que ocurrió en el primer período, se vaya diluyendo con el tiempo.

\begin{center}\includegraphics{grupo_prediccion_informe_files/figure-latex/unnamed-chunk-12-1} \end{center}

\begin{itemize}
\item
  La varianza del M2 es mayormente explicada por sí misma al principio,
  pero va perdiendo fuerza gradualmente, alcanzando aproximadamente la
  mitad de su explicación inicial a lo largo de 12 períodos.
\item
  En cuanto a la varianza de la Tasa Nominal de Cambio, esta se explica
  en menor medida por sí misma pero se mantiene más constante con el
  tiempo.
\item
  La varianza del IPC muestra una fuerte explicación por sí misma, que
  se mantiene por encima del 50\% a largo plazo.
\item
  Para la varianza de la TPM, se observa alrededor del 80\% de
  explicación al inicio, pero esta proporción se reduce
  considerablemente con el paso del tiempo.
\item
  Finalmente, en la varianza del PIB, se puede afirmar que es la que
  mantiene mayor consistencia a lo largo del tiempo en su explicación.
\item
  La descomposición de la varianza revela que una parte significativa de
  la variabilidad en M2 puede atribuirse a los choques en M2 mismo, así
  como a los choques en IPC y PIB, subrayando la relevancia de estas
  variables.
\item
  El modelo VAR nos ha permitido generar la siguiente predicción de la
  oferta monetaria basada en las variables macroeconómicas incluidas.
\end{itemize}

\begin{center}\includegraphics{grupo_prediccion_informe_files/figure-latex/unnamed-chunk-13-1} \end{center}

\begin{itemize}
\item
  Al analizar el gráfico, la línea azul representa la predicción
  mientras que el área sombreada indica el intervalo de confianza al
  95\%, que refleja la variabilidad real de la predicción.
\item
  En los primeros dos trimestres se observa un crecimiento, aunque con
  valores negativos indicando una posible disminución de la oferta
  monetaria entre un 1\% y un 5\%.
\item
  Desde el tercer trimestre hasta el séptimo, esta tendencia se mantiene
  constante, alcanzando un pico de casi 6\% de aumento, sugiriendo una
  continuación del crecimiento.
\item
  A partir del segundo año, específicamente en el octavo trimestre, es
  probable que se experimente una marcada desaceleración en el
  crecimiento de la oferta monetaria, lo que podría ser indicativo de un
  comportamiento cíclico típico de estas variables.
\item
  El horizonte de 8 trimestres se justifica debido a que más allá de
  este punto, la incertidumbre aumenta considerablemente. El área
  sombreada del intervalo de confianza se amplía y podría reflejar más
  el ruido en los datos, haciendo menos realista cualquier predicción a
  largo plazo. Se recomienda actualizar la información conforme pase el
  tiempo y mantener un horizonte de análisis de 8 trimestres para
  evaluar la persistencia del comportamiento cíclico, especialmente en
  respuesta a posibles shocks económicos a lo largo del tiempo. \newpage
\end{itemize}

\section{Conclusión}\label{conclusiuxf3n}

\subsection{Resumen de Resultados:}\label{resumen-de-resultados}

\begin{itemize}
\item
  Los resultados obtenidos confirman que las variables IPC y PIB tienen
  un impacto significativo en la predicción de la oferta monetaria M2.
  La hipótesis planteada se valida parcialmente, ya que, aunque algunas
  variables como la Tasa Nominal de Cambio (TCN) y la Tasa de Política
  Monetaria (TPM) no mostraron significancia estadística, el IPC y el
  PIB demostraron ser predictivos de M2, al menos a corto plazo.
\item
  Estos hallazgos están en línea con estudios previos que sugieren que
  la oferta monetaria está influenciada significativamente por variables
  económicas como los niveles de precios. La oferta monetaria M2 puede
  predecirse de manera efectiva utilizando un modelo VAR, y esta
  predicción puede refinarse con pruebas adicionales, como los análisis
  de impulso-respuesta y causalidad de Granger. Es esencial considerar
  si las predicciones son a corto o largo plazo, dependiendo de la
  relevancia para el tipo de análisis.
\item
  Es importante destacar los límites de las predicciones a muy largo
  plazo. Intentar pronosticar más allá de un horizonte razonable puede
  resultar en la dilución de la influencia de las variables
  explicativas, disminuyendo la precisión de los resultados obtenidos en
  los primeros períodos.
\item
  Respecto a la hipótesis, debe ser aceptada, dado que la predicción en
  este estudio se limita a un máximo de 8 trimestres (2 años). En este
  horizonte, las variables de mayor impacto son el IPC y el PIB, como lo
  sugieren los estudios mencionados. A largo plazo, el PIB va perdiendo
  fuerza predictiva, mientras que la TCN gana mayor relevancia junto con
  el IPC. Sin embargo, esta tendencia no es relevante para predicciones
  a corto plazo.
\item
  Se recomienda actualizar el modelo VAR con los datos más recientes a
  medida que estén disponibles. Esta actualización permitirá mantener la
  fiabilidad de las predicciones para los períodos más cercanos. El uso
  continuo de un horizonte de predicción de 8 trimestres asegurará que
  las predicciones sean precisas y útiles para la toma de decisiones
  económicas a corto plazo.
\end{itemize}

\subsection{Comparación de modelos.}\label{comparaciuxf3n-de-modelos.}

\begin{itemize}
\item
  Modelo ARD: Este modelo generalmente se utiliza para estudiar la
  relación a largo plazo entre una variable dependiente y otras
  explicativas.
\item
  Según los resultados, se mostró que no era significativo este modelo,
  revelando la variabilidad de los errores no es constante en el tiempo,
  que directamente afecta los coeficientes estimados.
\item
  Modelo VAR: Este modelo es útil principalmente cuando las variables
  pueden influenciarse mutuamente, en el contexto de variables
  macroeconómicas, esto es más plausible.
\item
  Según los resultados, la causalidad de Granger indica flujo de
  información desde el IPC, PIB y TCN hacia la M2, sin embargo la TPM
  no. Además la respuesta de M2 a sus propios choques, muestra un efecto
  inicial positivo que se va desvaneciendo con el tiempo.
\item
  El modelo VAR fue más adecuado para capturar la interrelación entre
  las variables, hay interdependencia significativa entre M2 y las
  variables explicativas escogidas, lo que nos permite realizar
  predicciones más precisas.
\end{itemize}

\section{Bibliografia}\label{bibliografia}

\begin{itemize}
\item
  Feldkircher, M., \& Tondl, G. (2020). Global factors driving inflation
  and monetary policy: A global VAR assessment. International Advances
  in Economic Research, 26(3), 225--247.
  \url{https://doi.org/10.1007/s11294-020-09792-2}
\item
  Informe de Política Monetaria Junio 2024 (LSE) - Rosanna Costa,
  Presidenta. (s/f). Banco Central de Chile. Recuperado el 3 de julio de
  2024, de
  \url{https://www.bcentral.cl/es/web/banco-central/contenido/-/details/ipom-junio-2024-lse-rosanna-costa}
\end{itemize}

\end{document}
